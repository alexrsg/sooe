\chapter{Orçamento}

O sistema deverá permitir:

\begin{itemize}
	\item Cadastrar orçamento;
	\item Definir a situação do orçamento:
	\begin{enumerate}
		\item estimativa;
		\item venda(inicial); e
		\item execução.
	\end{enumerate}
	\item Associar serviços ao orçamento.
\end{itemize}

Para cadastrar um novo orçamento é necessário que o projeto esteja em situação ``em aberto''.

Quando um orçamento tiver sua situação alterada para ``venda'', o projeto deverá ter sua fase alterada para ``fase de aprovação''.

Quando um orçamento de execução for definido, a fase do projeto deverá ser alterada para ``fase de execução''.

Um projeto de ``venda'' normalmente pode sofrer alterações e/ou aditivos (recursos ou serviços não previstos no orçamento de venda). Essas alterações devem ser registradas no orçamento de ``execução''. Já alterações quando de vontade do cliente devem ser feitas no orçamento de ``venda'' e \emph{\textbf{refletidas}} no orçamento de ``execução''.

\section{Atividade}

As atividades são definidas pela associação dos serviços a serem executados em um projeto através do orçamento. Os serviços são agrupados ao orçamento através de uma classificação de ``tipo de custo'', sendo filtrados por ``custo direto'', ``custo indireto'' e ``custo administrativo''. Essa atribuição pode ser definida pelo responsável, podendo futuramente definir outros tipos de custos dentro para um orçamento. Sendo assim, a apresentação dos serviços em um orçamento deve ser agrupada com base no tipo de custo definido. 

Uma atividade deve possuir:

\begin{itemize}
	\item Orçamento;
	\item Serviço; e
	\item Tipo de custo para o serviço no orçamento.
\end{itemize}

O sistema deverá permitir:

\begin{itemize}
	\item Associar/ Desassociar serviços de um orçamento; e
	\item Alterar o tipo de custo de um ou mais serviços da composição.
\end{itemize}

\section{Tipo de Custo}

O sistema deverá permitir:

\begin{itemize}
	\item Criar/Alterar/Remover tipo de custo; e
	\item Agrupar serviços em um orçamento através do tipo de custo.
\end{itemize}

O tipo de custo deve possuir:

\begin{itemize}
	\item Um código; e
	\item Uma descrição.
\end{itemize}
