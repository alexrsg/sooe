\chapter{Projeto}
Para a realização de uma obra são necessários alguns passos:

\begin{itemize}
	\item Contratação de escritório de arquitetura;
	\item Elaboração de ante-projeto de arquitetura;
	\item Elaboração dos projetos arquitetônicos;
	\item Aprovação do projeto legal na prefeitura;
	\item Contratação de escritório de projetos de estruturas e instalações;
	\item \emph{\textbf{Elaboração do orçamento da obra}};
	\item Elaboração do planejamento da obra; e
	\item Execução da obra.
\end{itemize}

Esses passos chamamos de projeto de obra ou somente projeto.

\begin{quote}
Um projeto é um esforço temporário empreendido para criar um produto, serviço ou resultado exclusivo. Os projetos e as operações diferem, principalmente, no fato de que os projetos são temporários e exclusivos, enquanto as operações são contínuas e repetitivas. \cite{PMBOKDP}
\end{quote}

Focamos aqui a terminologia de projeto para a área de engenharia civil. Para a engenharia civil, uma obra ou reforma é considerada como um projeto.

% São exemplos de projetos envolvidos em uma obra ou reforma:
% \begin{itemize}
% 	\item Projeto arquitetônico;
% 	\item Projeto estrutural;
% 	\item Projeto de hidráulica; e
% 	\item Projeto de elétrica.
% \end{itemize}

O orçamento de obra é uma das etapas de execução de uma obra, ou seja, um dos passos do projeto sendo normalmente é elaborado para um cliente em um local físico específico.