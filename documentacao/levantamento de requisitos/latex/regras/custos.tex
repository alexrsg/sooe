\chapter{Custos}

\section{Custo Direto}

Custo direto em um projeto é a junção de todos os recursos diretamente ligados às atividades (serviços) para a execução de um projeto.

Se uma atividade de \emph{Alvenaria}, por exemplo, podemos ter custos diretos:

\begin{itemize}
	\item Andaime;
	\item Alvenaria de blocos de concreto 19x19x39 cm;
	\item Alvenaria de blocos de concreto 14x19x39 cm; e
	\item Verga 10x15 cm em concreto armado;
\end{itemize}

Porém, para a execução dessa atividade, podemos ter a presença de custos indiretos como o custo de mão de obra de um funcionário para a montagem e desmontagem de Andaime, por exemplo, ou o custo de um funcionário para supervisão da atividade quando de grande duração.

\section{Custo Indireto}

Os custos indiretos são representados pelos coadjuvantes necessários à execução de uma atividade.

Alguns exemplos de custos indiretos que podem estar envolvidos como coadjuvantes à execução de atividades:

\begin{itemize}
	\item Gastos com salários de funcionários extras;
	\item Gastos com aluguéis;
	\item Gastos com telefone;
	\item Gastos com material de escritório;
	\item Gastos com energia elétrica;
	\item Gastos com propaganda;
	\item Gastos com acessoria técnica; e
	\item Gastos com combustíveis.
\end{itemize}