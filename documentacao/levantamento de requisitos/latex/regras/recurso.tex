\chapter{Recurso}

Recurso é tudo que representa unidade e que compõe as atividades (serviços) de um projeto. 

\section{Tipo de Recurso}

O recurso é subdividido em: 

\begin{itemize}
	\item Materiais;
	\item Mão de Obra;
	\item Equipamento; e
	\item Encargos.
\end{itemize}

\section{Unidade de Recurso}

Os recursos são medidos por unidades:

\begin{itemize}
	\item Kilograma (kg);
	\item Litro (l);
	\item Hora/Homem (h);
	\item Metro cúbico (M3);
	\item Metro quadrado (M2); e
	\item Outros.
\end{itemize}

O custo do recurso é dado por unidade, como demonstra a tabela \ref{tab:composicao}. Os valores são representados para uma única unidade de cada recurso, sendo que uma atividade poderá consumir \emph{n} quantidades de recursos.

Um exemplo para a atividade \emph{Alv. Bl. Concreto 9x19x39} para construção de uma parede de 8m x 3m (24 m2). Assim, de acordo com a tabela de composição do serviço \ref{tab:composicao}, o serviço \emph{Alv. Bl. Concreto 9x19x39 vedação} tem um custo de 39,91 por m2. Para a execução da parede utilizando a atividade acima, seriam necessárias 24 unidades, totalizando seu custo em 957,84.