\chapter{Composição}

Composição é o nome que se dá à junção de recursos desprendidos em uma determinada atividade, é o vinculo entre serviço, recurso e outros serviços alocados para o mesmo.

Por exemplo, uma atividade de \emph{Alv. Bl. Concreto 9x19x39 vedação} teria como composição os seguintes items:

\begin{table}[h]
	\centering
	\begin{tabular}{|l|l|c|r|}
	\hline
	Código&			Descrição&						Unidade&		Valor 	\\ \hline
	3.36.020&		bloco concreto 9x19x39&			pç&				13,13	\\ \hline
	3.05.300&		Arg. Serrana F11 saco 40kg&		kg&				17,78	\\ \hline
	2.10.020&		Pedreiro&						h&				 1,00	\\ \hline
	2.10.050&		Servente&						h&				 1,00	\\ \hline
	\end{tabular}
	\caption{Composição de atividade \emph{Alv. Bl. Concreto 9x19x39 vedação}}
	\label{tab:composicao}
\end{table}

Cada item demonstrado acima é considera um recurso.

Para realizar determinado serviço serão necessários usar recursos estipulados e outros serviços. Essa listagem do "que" tem que ser feito para fazer um serviço se da o nome de composição. Ex., para fazer o recurso ``Cobertura em telhas onduladas, sem amianto, com espessura de 4mm, fixadas por pregos, inclusive vedacao, exclusive o madeiramento, Vogatex ou similar. Fornecimento e colocacao.'' será necessário usar os seguintes recursos:

\begin{itemize}
	\item 3\% incidente sobre mao de obra direta com Encargos Sociais para cobrir despesas de EPI e ferramentas;
	\item Conjunto de vedacao para telha ondulada (arruela galvanizada com borracha)
	\item Prego com cabeca, de (18x30);
	\item Telha ondulada sem amianto, com espessura de 4mm, medindo: (2,44x0,50)m, Vogatex ou similar;
	\item Carpinteiro - forma de concreto;
	\item Servente Tributos sobre o faturamento (7.56\%);
\end{itemize}

E o serviço de por exemplo, fixar colunas.